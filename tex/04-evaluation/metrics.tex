\documentclass[../../report.tex]{subfiles}
\begin{document}
\section{Metrics}

In each step, the model chooses one of 38 possible events. This means that
random guessing would give an overall accuracy of \(\frac{1}{38} \approx 3\%\).
However, the model is quick to learn that \emph{no-event} is by far the most
common label in the dataset, leading to a rapid increase in \emph{no-event
accuracy}. This can give a false impression of good performance -- in reality,
the model still has no ability to predict notes correctly.

\missingfigure{Initial accuracies}

In order to avoid the issue explained above, all experiments were evaluated
according to the single metric of \emph{event accuracy}. More precisely, this is
the percentage of correct predictions in positions whose ground truth label is
\emph{note-off} or any of the 36 \emph{note-on} events.

For LSNNs, it is also interesting to keep track of the average neuron firing
rate. Firing rate regularisation\todo{ref} was applied in all performed
experiments, and the network was able to converge to the target rate. Overall,
this means that the model learns to represent information using a temporal
coding. This is especially important on neuromorphic hardware such as SpiNNaker
\cite{Furber2014}, where a reduction in firing rate leads to reduced energy
consumption.

\missingfigure{Firing rate graph}

\end{document}
