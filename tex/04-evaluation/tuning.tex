\documentclass[../../report.tex]{subfiles}
\begin{document}
\section{Hyperparameter Tuning}

In order to ensure a fair comparison between the two different model
architectures, it is imperative to ensure that the values of hyperparameters are
optimal. Generally, this is a non-trivial task, and in research projects it
often accounts for the vast majority of compute time \cite{Strubell2019}. Given
the hardware limitations in this project, hyperparameter search was limited to a
handful of informed guesses, as opposed to a more systematic approach (e.g. grid
search).

\subsection{LSNN parameters}

By analysing successful LSNN experiments described in literature
\cite{Bellec2018LSNN, Bellec2020}, a base configuration was devised (table
\ref{tab:lsnn-base}).

\begin{table}
  \begin{center}
    \renewcommand{\arraystretch}{1.25}
    \begin{tabular}{ r | r | l  }
      Parameter & Value & Description
      \\ \hline
      \(B\) & 128 & Batch size
      \\
      \(\eta\) & 0.001 & Learning rate
      \\
      \(\gamma\) & 0.3 & Pseudo-derivative dampening factor
      \\
      \(N\) & (512) & Neurons per layer (tuple)
      \\
      \(n_\mathrm{in}\) & 1 & Neurons per input class
      \\
      \(n_\alpha\) & 0.4 & Fraction of neurons with adaptation
      \\
      \(\delta t\) & 1.0 & Time step (ms)
      \\
      \(t_r\) & 2 & Number of refractory time steps
      \\
      \(t_\mathrm{in}\) & 5 & Number of time steps per input event
      \\
      \(\tau_\mathrm{out}\) & 3.0 & Output readout time constant (ms)
      \\
      \(\tau_v\) & 20.0 & Membrane potential time constant (ms)
      \\
      \(\tau_\alpha\) & 500.0 & Neuronal adaptation time constant (ms)
      \\
      \(\alpha\) & 0.1 & Neuronal adaptation magnitude
      \\
      \(v_\mathrm{thr}\) & 1.0 & Threshold membrane potential
      \\
      \(f_\mathrm{in}\) & 1000 & Input firing rate (Hz)
      \\
      \(f_\mathrm{reg}\) & 20 & Firing rate regularisation target (Hz)
      \\
    \end{tabular}
  \end{center}
  \caption{Base LSNN configuration}
  \label{tab:lsnn-base}
\end{table}

\subsection{LSTM parameters}


\end{document}
