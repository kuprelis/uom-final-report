\documentclass[../../report.tex]{subfiles}
\begin{document}
\chapter{Implementation}

Having laid out all the prerequisite concepts in the previous chapter, let's now
recap the scope of this project. Basic RNN, Magenta's LSTM-based melody
generation model, shall be used as the starting point. The aim is to develop a
variant of this model that uses an LSNN instead. Then, a melody dataset shall be
compiled from a set of MIDI files. These melodies will be used to train and
evaluate equivalent configurations of both models. Finally, it will be possible
to generate new melodies and to assess their musicality.

The code for this project is written in Python. It uses TensorFlow, an
open-source machine learning library. This was an obvious choice, as the
codebase of Magenta is based purely on TensorFlow.

\subfile{architecture}
\subfile{preparation}

\end{document}
