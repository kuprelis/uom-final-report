\documentclass[../../report.tex]{subfiles}
\begin{document}
\section{Melody Generation}

Once the model is trained to a desired extent, it can be used to generate
melodies. By default, a single C4 note is used as the starting point.

A naive generation algorithm would simply select the event with the highest
probability at each step, feed it back into the model as input, and repeat until
a sufficiently long sequence is produced. However, a fully greedy strategy is
unlikely to produce a sequence that has the highest overall likelihood. Consider
the following toy example, where \(\bm{p}\) represents the maximal transition
probabilities observed when following a fully greedy strategy, and \(\bm{q}\) is
obtained by a strategy that instead picks the second best option in the initial
step.

\begin{align*}
  \bm{p} &= [0.9, 0.1, 0.1] & \prod_{i} p_i &= 0.009
  \\
  \bm{q} &= [0.5, 0.5, 0.5] & \prod_{i} q_i &= 0.125
\end{align*}

As made evident by \(\bm{q}\), it is sometimes worthwhile to pick a less likely
option, if it leads to higher subsequent probabilities.

\subsection{Beam search}

Instead of using the naive strategy, melodies are generated using \emph{beam
search}. Strictly speaking, it is also a greedy strategy. However, beam search
keeps more options in consideration for a longer time, which is good enough to
find sequences with considerably higher likelihoods.

\subsubsection{Algorithm}

Beam search is parameterised by a branch factor \(b\) and beam width \(w\),
where the beam is a list of potential output sequences. The algorithm iterates
until the sequences reach a specified length, repeating two main steps in each
iteration:

\begin{enumerate}
  \item \textbf{Branching}. For each sequence in the beam, the model is used to
  predict a probability distribution over the next events. The top \(b\) events
  are used to branch each sequence into \(b\) extended sequences.

  \item \textbf{Pruning}. The sequences in the current beam are sorted by
  likelihood. The top \(w\) sequences are kept, while others are discarded.
\end{enumerate}

Finally, the most likely sequence in the beam is returned as the output.

\subsection{Agency}

Generative models are most useful when their output obeys conditioning signals.
For example, a musician may wish to create a melody that begins with a specific
motif, with the rest dreamed up by the model. To enable the user with such
agency, this model exposes two customisation parameters.

\subsubsection{Primer melody}
The model can be primed with an input melody, given either as a MIDI file, or as
an array literal on the command line. This melody is then fed into the model to
initialise its state. Next, the generation process continues as before,
hopefully producing a sensible continuation to the primer melody.

\subsubsection{Temperature}
Before applying the softmax function to the output layer, activations are
divided by the \emph{temperature} value, which defaults to 1. Increasing this
value makes the probability distribution smoother, introducing more randomness
into the output. Conversely, decreasing the temperature accentuates the peaks in
the probability distribution, leading to more predictable behaviour.

Ironically, using beam search usually leads to `melodies' consisting of nothing
but silence. After all, it is quite likely that the \emph{no-event} label
dominates the dataset. Nevertheless, this problem can be alleviated by
increasing the temperature. This makes beam search less biased towards silence.

\end{document}
