\documentclass[../../report.tex]{subfiles}
\begin{document}
\section{Objectives}

We approach melody generation as a many-to-many classification task, where the
sequence of preceding notes is used to produce a probability distribution over
all notes at each position of a melody. Such a distribution shall be learned
from a large dataset of melodies. This results in the following set of key
tasks:

\begin{enumerate}
  \item Implementing an LSNN-based melody generation model.
  \item Acquiring training melodies.
  \item Encoding melodies into spike-based inputs.
  \item Training multiple LSNNs in search of optimal hyperparameters.
  \item Training an LSTM configuration known to give good results.
  \item Comparing the best performing LSNN with the LSTM.
  \item Analysing the musicality of generated melodies.
\end{enumerate}

This project does not aim to improve the state of the art, therefore the success
criterion is the mere completion of the above tasks.

\end{document}
