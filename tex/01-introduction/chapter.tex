\documentclass[../../report.tex]{subfiles}

\begin{document}
\chapter{Introduction}

Music generation is a well established domain of machine learning, dating at
least as far back as 1989. Since then, many successful implementations have made
use of general sequence modelling techniques such as RNNs, LSTMs, and more
recently, Transformers. Due to the high power and memory requirements of these
models, there is an increasing amount of interest in biological neural networks,
which are intrinsically power-efficient and sparsely connected. Nevertheless,
biological networks exhibit astounding cognitive capabilities, which makes them
particularly relevant in the quest to alleviate the aforementioned issues of
artificial neural networks.

\section{Motivation}
Music is considered to be a cultural universal \cite{Mehreaax0868}.

\section{Objectives}
\section{Pandemic impact}
At the time of writing, the COVID-19 pandemic is ongoing. Unfortunately, it has had a negative impact on this project. Due to the introduction of a national lockdown during the first semester, it was necessary to adapt to working from home, which disrupted my focus and productivity.

\end{document}
