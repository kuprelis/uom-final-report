\documentclass[../../report.tex]{subfiles}
\begin{document}
\section{Overview}

Music generation is a well established domain of machine learning, with the
first RNN-based approach dating back to 1989 \cite{Todd1989}. Since then, many
successful implementations have piggybacked on advancements in general sequence
modelling techniques, such as Long Short-Term Memory \cite{Eck2002}, attention
\cite{Waite2016}, and more recently, the Transformer \cite{Huang2018}. However,
due to the high power and memory requirements of these model architectures,
there is an increasing amount of interest in biological neural networks, which
are intrinsically power efficient and sparsely connected, all the while
exhibiting astounding cognitive capabilities. This makes them particularly
relevant in the quest to reduce the resource requirements of artificial neural
networks.

Nevertheless, there is good reason why biologically inspired spiking neural
networks (SNNs) are still dominated by artificial neural networks (ANNs). SNNs
carry out their computations via spikes, which are inherently
non-differentiable. This complicates the use of backpropagation, which is the
basis of many successful ANN optimisation techniques. In the case of networks
with recurrent connections, like those found in the brain, it is common to use a
variant known as backpropagation through time (BPTT). Approximating the spike
gradient with a pseudo-derivative has been found to give good results
\cite{Bellec2018LSNN}, but the problem is that BPTT requires storing previous
states of each neuron, which is not biologically realistic \cite{Lillicrap2019}.

Although promising solutions to the SNN training dilemma have emerged
\cite{Bellec2020}, as we find in this project, it is difficult to achieve good
performance on general purpose computers. One could argue that this comparison
is somewhat unfair, as the past few years have seen significant effort to
develop highly optimised machine learning frameworks geared towards ANNs.
Indeed, SNN simulations can likewise be optimised through the use of specialised
neuromorphic hardware, like our department's own SpiNNaker supercomputer
\cite{Furber2014}. Still, utilising such platforms is often much more involved
than an ML framework, so for the time being, efficient SNNs remain reserved for
those willing to take a deep dive. Given that efficiency was not a concern in
this project, all simulations were performed using GPU-accelerated SNN
implementations on general purpose computers.

Let's now turn back to our goal of music generation. If we are to use neural
networks to generate melodies, we must ensure that the network has the
capability to capture complex melodic patterns, or more generally, long-term
sequential dependencies. This is exactly the promise of an LSTM, and that is why
it has been so popular for the task of music generation. It has been shown that
SNNs can acquire similar properties through the inclusion of neuronal adaptation
\cite{Bellec2018LSNN}. Such spiking networks have been named LSNNs, due to their
similarity to LSTMs. This similarity also gives hope to solve the task of
sequence modelling, and so that is the hypothesis to be tested in this project.

\end{document}
