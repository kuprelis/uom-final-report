\documentclass[../../report.tex]{subfiles}
\begin{document}
\section{Motivation}

Music is considered to be a cultural universal \cite{Mehr2019}. That is to say,
it is a common trait across all known human cultures. Moreover, music has been a
part of the human condition throughout all of recorded history, so it is natural
that as time goes on and technology develops, newfound knowledge is sooner or
later applied in music-making. Just like the people of Ancient Greece used their
early understanding of hydraulics to build the water organ, it was not long
after the advent of electricity that we saw the invention of the Telharmonium,
the first electronic organ. And now, as we witness the proliferation of machine
learning applications, it only makes sense that we look into leveraging these
intelligent machines to make music.

Having attended music school prior to university, I am particularly keen on
combining my hobby of music-making with my passion for computer science. When it
was time to pick a final year project, it felt natural to explore the
intersection of these two interests of mine. To put a further twist on it, I
decided to take advantage of our department's active research into spiking
neural networks. And so, this project came to be.

\end{document}
