\documentclass[../../report.tex]{subfiles}
\begin{document}
\section{Future Work}

Spiking neural networks and music generation are both complex problem domains on
their own, and their combination is even more so. By the end of this project, it
had become clear that there is still much to be explored at the intersection of
these two active research fields. In this concluding section of the report, we
shall outline some ideas that could seed future works.

\subsection{Neuromorphic implementation}

During training, it was observed that the LSNN is around an order of magnitude
slower. Potentially, an efficiency improvement could be achieved through the use
of neuromorphic hardware. This approach comes with its own set of challenges,
such as limited synaptic connectivity or limited memory \cite{Liu2018}.
Fortunately, it has been shown that enforcing sparse connectivity only leads to
a minor loss in accuracy on several standard benchmarks
\cite{Bellec2018Rewiring}. This, along with results achieved by the e-prop
online learning algorithm \cite{Bellec2020}, gives hope that our melody
generation model could also be implemented on neuromorphic hardware.

\subsection{Attention}

Today, some of the most successful results in symbolic music generation
\cite{Huang2018} are achieved through the use of Transformers, which rely
exclusively on attention mechanisms \cite{Vaswani2017}. It remains to be seen
how the concept of attention translates to the domain of biologically inspired
neural networks, and how it could improve melody generation.

\subsection{Frequency domain}

In the human ear, different parts of the cochlea resonate at different
frequencies, effectively performing a mechanical version of the Fourier
transform. This means the human auditory system operates in the frequency
domain, so it would be interesting to explore the possibilities of SNN music
generation with spectral inputs.

\end{document}
